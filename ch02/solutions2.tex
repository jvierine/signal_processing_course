\newpage
\section{Suggested solutions: Python}


\begin{enumerate}
\item If you use Linux type: \verb|git clone https://github.com/jvierine/signal_processing| in the terminal and hit enter.
On Windows and MacOS, I don't know...? Google it, I guess. You can also just download it directly from GitHub.

\item 
\begin{enumerate}[a)]
\item Go here \url{http://numpy.org} to read documentation. 
\item Reading the documentation you'll find that \verb|n.arange| creates a NumPy array 
      of evenly spaced numbers from 0 to the provided number minus 1. In the case of the code here, 
      the \verb|n.arange| function will sequence the numbers from 0 to 99 and return a NumPy array containing these numbers. 
      See documentation: \url{https://numpy.org/doc/stable/reference/generated/numpy.arange.html}.
      The documentation for \verb|n.exp| can be found here \url{https://numpy.org/doc/stable/reference/generated/numpy.exp.html}.
      The function simply computes $e^{x}$, where $x$ can be real or complex, or even a NumPy array. If so, a NumPy array is returned. 
\item Listing \ref{circle_plot} now plots a circle. 
\lstinputlisting[language=Python,caption=Code is modified to plot a circle,label=circle_plot,linerange={0-25}]{ch02/code/ex2_2.py}

\begin{marginfigure}
\includegraphics[width=\textwidth]{ch02/figures/circle_plot.png}
\caption{Output of Listing \ref{circle_plot}}
\end{marginfigure}

\item The unit circle can be described by the equation $x^{2} + y^{2} = 1$, which is satisfied by $x=\cos(t)$ and $y =\sin(t)$. 
\end{enumerate}


\item Listing \ref{ipi} shows how to compute and print $e^{i\pi}+1$ in Python. The resulting output is \verb|1.2246467991473532e-16j|, which is not zero.
The reason for this is due to floating point errors. The value is very close to 0, but not exactly 0. 

\lstinputlisting[language=Python,caption=Computing $e^{i\pi}+1$, label = ipi]{ch02/code/ex2_3.py}

\item The following is a solution for Exercise 4.

\begin{enumerate}[a)]
\item Running the code generates the plot shown in the exercise. 

\item Adding comments in Python can be done using \verb|#|, Listing \ref{comments_on_code} shows some comments.
\lstinputlisting[language=Python,caption=Commented Python code, label=comments_on_code]{ch02/code/ex2_4.py}

\item One can specify the data type of a NumPy array. The data types supported can be found in the documentation. 
These include: \verb|complex64|, \verb|complex128|, \verb|float32|, \verb|int32|, among others. 
Pros and cons with the \verb|complex64| and \verb|complex128| include: size and accuracy. The \verb|complex64|
takes less memory, but is less accurate, while \verb|complex128| can store more accurate numbers, but takes more memory.

\end{enumerate}





\end{enumerate}
