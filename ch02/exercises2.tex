\newpage
\section{Exercises}

These exercises are intended to get you started with programming with
\emph{\index{Python}{Python}} and the \emph{\index{NumPy}{NumPy}} and
Matplotlib libraries. If you are already familiar with these topics,
you may want to skip these exercises.

\begin{enumerate}
 \item Obtain the source code for the examples by cloning the GitHub
   repository for this course. 
\item The Hello World program shown in Listing \ref{lst:pythonhw} plots a complex sinusoidal signal. 
  \begin{enumerate}[a)]
  \item Go on the NumPy website \url{http://numpy.org}, and find the documentation of the functions included in the package.
  \item Use this documentation to determine what \verb|numpy.arange|, and \verb|numpy.exp| do. 
  \item Modify the code so that it plots a circle on the complex
    plane, by plotting the real part of the complex sinusoidal
    signal \verb|csin| on the x-axis, and the imaginary part on the
    y-axis.
  \item Explain why the real and imaginary component draw a circle on the complex plane.
  \end{enumerate}

\item Use Python to calculate and print out the value of $e^{i \pi} + 1$. Note that $i$ in Python, is denoted
  with \verb|1j|.
  
\item The use of built-in Numpy functions will let you program
  efficiently. The program shown in Listing \ref{lst:exercise002}
  demonstrates the use of Numpy functions to evaluate the Mandelbrot
  set with a slight twist.

\lstinputlisting[language=Python,caption={\texttt{001\_hello\_world/mystery.py}},label=lst:exercise002]{code/001_hello_world/mystery.py}

\begin{enumerate}[a)]
  \item Run the program shown in Listing \ref{lst:exercise002}. You
    	should see a plot like the one shown in Figure \ref{fig:mandelbrot}.
  \item Describe what each line of the program does by adding comments to the code.
  \item It is possible to specify the data type of any NumPy array
      	using the \verb|dtype| attribute? There are two complex valued
      	datatypes available in NumPy: \verb|complex64| and \verb|complex128|. 
      	What are the pros and cons of using the \verb|complex64| datatype 
      	instead of the \verb|complex128| datatype? 
\end{enumerate}


\begin{figure}
\includegraphics[width=0.9\textwidth]{ch02/figures/mystery.png}
\caption{The Mandelbrot set example demonstrates the use of NumPy array functions and complex numbers. The plot
  shows the phase of complex number $z_{12}$ after 12 iterations of
  $z_{n+1} \leftarrow z_n^2 + c$, starting with $z_0 = 0$.}
\label{fig:mandelbrot}
\end{figure}



\end{enumerate}
