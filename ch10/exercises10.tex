\newpage
\section{Exercises: Linear Time-invariant Systems}
\begin{enumerate}

  % Exercise 1
  \item Prove that the convolution operation is commutative.
        That is, show that $a[n]*b[n]=b[n]*a[n]$. In addition, show that
        this holds true for the continuous-time convolution operation.

  % Exercise 2
  \item A running average system is defined as:
        \begin{equation}
          y[n] = \mathcal{T}\{x[n]\} = \frac{1}{L}\sum_{k=0}^{L-1} x[n-k]
        \end{equation}

        \begin{enumerate}[a)]
          % Exercise 2a)
          \item Show that the running average system is a linear time
                invariant system using the test for linearity and time
                invariance.
          % Exercise 2b)
          \item What is the impulse response $h[n] =
                  \mathcal{T}\{\delta[n]\}$ of the running average system?
          % Exercise 2c)
          \item How many non-zero values does the impulse response $h[n]$ have?
          % Exercise 2d)
          \item We feed a discrete-time complex sinusoidal signal
                $x[n]=e^{i\hat{\omega}_0 n}$ into the system. Show that the
                output is of the form $y[n]=A e^{i\phi} e^{i\hat{\omega}_0 n}$,
                in other words, a discrete-time complex sinusoidal signal with
                the same frequency as the input signal.
          % Exercise 2e)
          \item Continue with task d). Let us assume that
                $L=4$. What is the amplitude of the output signal $A$ when we
                have a normalized angular frequency of:
                \begin{enumerate}
                  \item $\hat{\omega}_0 = 0$ radians per sample?
                  \item $\hat{\omega}_0 = \pi$ radians per sample?
                  \item $\hat{\omega}_0 = 0.5\pi$ radians per sample?
                \end{enumerate}

          % Exercise 2f)
          \item Show that system $y_2[n]=\mathcal{T}\{ \mathcal{T}\{x[n]\} \}$ is also an LTI system.
          % Exercise 2g)
          \item What is the impulse response $h_2[n]=\mathcal{T}\{\mathcal{T}\{\delta[n]\}\}$?
                Sketch a plot of the non-zero values $h_2[n]$.
        \end{enumerate}

  % Exercise 3
  \item Consider the following running average filter:
        \begin{equation}
          y[n] = \frac{1}{L}\sum_{k = 0}^{L - 1}x[n - k].
          \label{eq:ex:filter}
        \end{equation}

        \begin{enumerate}[a)]
          % Exercise 3a)
          \item Let $x[n]$ be the signal:
                \[
                  x[n] = \begin{cases}
                    1.2, \quad n = 0, \\
                    4.3, \quad n = 1, \\
                    4.7, \quad n = 2, \\
                    3.3, \quad n = 3, \\
                    2.9, \quad n = 4.
                  \end{cases}
                \]
                Compute the resulting signal $y[n]$ from using the filter in \ref*{eq:ex:filter}
                by hand, using $L = 3$. Values of $n$ for which $n \notin \{0, 1, 2, 3, 4\}$
                are treated as $0$.

          % Exercise 3b)
          \item Implement a function \verb|running_average_filter(L, x)| that computes the running average filter
                for a given $L$ and signal $x$. Apply the filter on the signal $x$ from the previous exercise
                to verify that you get the same answer (up to some rounding errors).
          % Exercise 3c)
          \item The code shown in Listing \ref*{ch10:precode1} makes plots of a noisy signal
                along with the filtered signal, using the running average filter from Equation \ref*{eq:ex:filter}.
                Add the function you wrote from the previous exercise to the code, and verify that the noisy
                signal has been smoothened by the filter. It should look similar to Figure \ref*{fig:avg_filter}.
                \lstinputlisting[language=Python, caption=Given code for exercise 3c, label=ch10:precode1]{ch10/code/starting_code1_ex3c.py}
          % Exercise 3d)
          \item Download \verb|7na.wav| from the course GitHub examples, 
                \url{https://github.com/jvierine/signal_processing/tree/master/018_reverb}. 
                Apply the running average filter in Equation \ref{eq:ex:filter} to the audio using 
                $L = 4, 25, 50, 100$. Can you detect with your ear what happens to the
                low and high frequency components of the signal as a result of
                the operation?
        \end{enumerate}

  % Exercise 4
  \item Obtain the demonstration code that implements a reverb effect
        using a convolution operation:
        \begin{equation}
          y[n] = h[n]*x[n],
        \end{equation}
        where $x[n]$ is the input signal, $h[n]$ is the impulse response
        of a room, and $y[n]$ is the output signal with the reverb
        effect. Download \verb|7na.wav| from the course GitHub examples,
        \url{https://github.com/jvierine/signal_processing/tree/master/018_reverb}
        or alternatively use your own audio file.

        \begin{enumerate}[a)]

          % Exercise 4a)
          \item Run the example code and verify that the filtered signal indeed sounds like it
                is played in a large room by playing \verb|7na.wav| and the file \verb|reverb.wav|
                produced by the script.

          % Exercise 4b)
          \item Find the part in the code where a convolution between the audio
                signal and the impulse response is evaluated. Plot the impulse
                response of the FIR filter applied to the input signal.

          % Exercise 4c)
          \item Figure out how to increase and reduce the amount of reverb, i.e.,
                to make it sound like the audio signal is played in a large or small
                room with many surfaces that reflect sound waves. What does the
                impulse response for a large room and a small room look like?

          % Exercise 4d)
          \item Now try to create an impulse response of an echo from a distance
                of 1000 meters using the following impulse response
                $h[n] = \delta[n] + 0.5\delta[n-n_0]$. Assuming that the propagation velocity of
                sound is 343 $\frac{\mathrm{m}}{\mathrm{s}}$, determine 
                a suitable value for $n_0$.
        \end{enumerate}

\end{enumerate}