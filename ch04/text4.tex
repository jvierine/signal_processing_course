% what is a signal and what is a system?
%
% introduces:
% - signal
% - continuous-time and discrete-time signal
% - dependent and independent variable
% - system
%
\newthought{What is a signal and what is a system?} By
a \emph{\index{signal}{signal}}, we mean an information carrying mathematical function. Any function that has a value as a function of one or more variables is in essence a signal. 
By a \emph{\index{system}{system}}, we denote a mathematical operation that modifies a signal. A system consists of the precise mathematical description of how a signal fed into a system is modified by the system to produce an output signal.

The definition of signals and systems are merely abstract concepts, which have gained acceptance in the engineering community. In the context of mathematics, signals could also be called functions or vectors. 
Systems would be called functions or operators. In computer science, signals are often treated as arrays of numbers in the memory of a computer and systems are algorithms and computer programs that operate on these arrays.

\section{Signals}
\index{signal}

\begin{marginfigure}%
\begin{center}
\begin{tikzpicture}
\begin{axis}[width=\textwidth,
	xticklabels=\empty,
	yticklabels=\empty,		
	xmin=-1.5,xmax=6,
	axis x line=bottom,
	axis y line=left,
	xlabel={Independent variable $t$},
	xlabel style={below},
	ylabel={Dependent variable $x(t)$},
    xlabel style={ yshift = { 1em } },
    ylabel style={ yshift = { -2.2em } }
]
\addplot[draw=blue,domain=-1:7,samples=150] {(x>0)*exp(-x)};
\end{axis}
\end{tikzpicture}
\end{center}
\caption{Continuous-time signal.}
\label{fig:ctfig}
\end{marginfigure}

\begin{marginfigure}%
\begin{center}
\begin{tikzpicture}
\begin{axis}[width=\textwidth,
%	title={Discrete-time signal},
	xticklabels=\empty,
	yticklabels=\empty,	
	xmin=-1.5,xmax=6,
	axis x line=bottom,
	axis y line=left,
	xlabel={Independent variable $n$},
	xlabel style={below},
	ylabel={Dependent variable $x[n]$},
    xlabel style={ yshift = { 1em } },
    ylabel style={ yshift = { -2.2em } }
]
\addplot+[ycomb,domain=-1:5,samples=15] {(x>0)*exp(-x)};
\end{axis}
\end{tikzpicture}
\end{center}
\caption{Discrete-time signal.}
\label{fig:dtfig}
\end{marginfigure}


\newthought{A \index{signal}{signal}} is a mathematical function $x(t)$, which describes the value of a \index{dependent variable} dependent variable $x$, as a function of an \index{independent variable}independent variable $t$. An independent variable ``sweeps'' through all possible values. The dependent variable is the variable that changes as a function of the independent variable and conveys information.

Here's an example. When describing electric potential as a function of time $V(t)$, time $t$ is the independent variable and the electric potential $V(t)$ as a function of time is the dependent variable. Time by itself does not convey information, but electric potential as a function of time does.

Physical ``real world'' signals are modeled as continuous-time signals. Examples include, amongst countless others:
\begin{itemize}
 \setlength\itemsep{0.25em}        
\item temperature,
\item density,
\item pressure as a function of time and space (sound, seismic waves),
\item electric field as a function of time and position
  (electromagnetic waves), or
\item electrical current in a circuit.
\end{itemize}
Physics relies on differential calculus, with integration and differentiation as elementary operators. As we will later see, differential calculus can be studied through methods of signal processing, especially the spectral techniques and the Fourier transform are useful tools that can be applied in differential calculus.

This course will focus primarily on one-dimensional signals. These signals are complex valued functions $x(t) \in \mathbb{C}$ of a real valued argument $t\in \mathbb{R}$. This will naturally also cover the
special case, where the signal is real-valued
$x(t) \in \mathbb{R} \subset\mathbb{C}$. 

By convention, we will refer to the independent variable of a signal as time, even though this variable doesn't necessarily have to indicate time. It can represent anything. For example, the independent variable can just as well be, e.g., distance.

Signals can be continuous or discrete. Following a commonly adopted practice, we will use round brackets for continuous-time signals (e.g., $x(t)$) and square brackets for discrete-time signals (e.g., $x[n]$). 

In the case of discrete-time signals, the sample index
$n \in \mathbb{Z}$ is unitless. A discrete-time signal is merely a sequence of numbers. The only way to associate meaning to this sequence of numbers is the a priori knowledge of how the signal was discretized. This allows us to, e.g., map the $n$th sample to a real valued time.

An example of a continuous-time and a discrete-time signal is shown in Figures \ref{fig:ctfig} and \ref{fig:dtfig}. When plotting signals graphically, it is customary (but not mandatory) to use the horizontal axis for the independent variable, and the dependent variable on the vertical axis.

To summarize, the two main types of signals that this course deals with are one-dimensional continuous-time $x(t)$ and one-dimensional discrete-time signals $x[n]$. Continuous-time signals are mappings from the real axis (time) to the set of complex numbers:
\begin{equation}
\boxed{
x: \mathbb{R} \rightarrow \mathbb{C}
}
\end{equation}
Discrete-time signals are mappings from the set of integers to the set of complex numbers:
\begin{equation}
\boxed{
x: \mathbb{Z} \rightarrow \mathbb{C}
}
\end{equation}
Signal processing of higher dimensional signals are essentially functions of the form
\begin{equation}
x: \mathbb{R}^N \rightarrow \mathbb{C}
\end{equation}
or in the case of discrete-time:
\begin{equation}
x: \mathbb{Z}^N \rightarrow \mathbb{C}
\end{equation}

\newthought{The first experimental detection of gravitational waves} using the Laser Inteferometer Gravitational-Wave Observatory (LIGO) is shown in Figure \ref{fig:ligo_meas}. This signal is an example of a one-dimensional signal. The signal is thought to be caused by two black holes with masses around 30 solar masses merging together, 1.3 billion light-years from Earth. The independent variable on the x-axis is time, and the dependent variable is strain (stretching) of space that occurs due to a gravitational wave passing through the instrument. This is measured by comparing the relative lengths of two 4 km long laser interferometer arms. 

\begin{figure}
\begin{center}
\includegraphics[width=\textwidth]{ch04/figures/dc_fg.png}
\end{center}
\caption{Two independent one dimensional signals measured by LIGO depicting strain. This is stretching of space due to a gravitational wave passing through two geographically separated laser interferometer observatories. One in Hanford, WA (left), and one in Livingston, LA (right). The figure is from: B. P. Abbott et al. (LIGO Scientific Collaboration and Virgo Collaboration) Phys. Rev. Lett. 116, 061102 – Published 11
February 2016.}
\label{fig:ligo_meas}
\end{figure}

\newthought{Signals can be of arbitrary dimension}. For example, an image is a 2d signal, where the dependent variable is intensity $I(x,y)$ measured as a function of two independent spatial variables $x$ and $y$ that indicate distance from origin along two orthogonal axes.  An example of a discrete-time two-dimensional signal is shown in Figure \ref{fig:bh_example}. 
It represents an image of the emission from hot gas in the event horizon of a black hole around the M87 galaxy. 
The image is obtained using a technique called very long baseline interferometry, which utilizes measurements from radio telescopes around the world. 
These measurements are combined to simulate a large telescope with the resolution equivalent to a telescope approximately the size of Earth.\footnote{The Fourier transform, which is one of the central themes in this course, is a key part of the mathematics of very long baseline interferometric imaging.}

Even higher dimensional signals can be used. Consider for example a video. It is a signal that contains the intensity of an image as a function of time. You can think of a moving picture (movie) as a three-dimensional signal: $I(x,y,t)$, where the intensity of the image is a function of two-dimensional position as well as time.

While we will focus primarily on one dimensional signals in order to keep things simple, many of the concepts we discuss can be generalized to multiple dimensions. 
Two-dimensional signal processing is called image processing, and it shares many of the same basic concepts with one dimensional signal processing. 
I will try to occasionally give examples of signal processing with higher dimensional signals.

%\begin{figure}
%\begin{center}
%\includegraphics[width=\textwidth]{ch01/2dsig.png}
%\end{center}
%\label{fig:2dexim}
%\caption{An example of a 2d signal: 32-cm wavelength opposite
%  circularly polarized inverse synthetic aperture radar map of the
%  Moon, obtained using the EISCAT radar in Troms\o{}. Image intensity
%  represents backscattered power from the surface of the Moon.}
%\end{figure}

\begin{figure}
\begin{center}
\includegraphics[width=0.68\textwidth]{code/004_dft_2d/bhi.png}
\end{center}
\caption{An example of a 2d signal: A Very Long Baseline Interferometric radio image of
the black hole at the center of galaxy M87. The intensity depicts what is thought to be hot gas outside the event horizon of the black hole. Credit: Event Horizon Telescope collaboration et al. 2019.}
\label{fig:bh_example}
\end{figure}

\newthought{Digital signals} on a computer, are also quantized. This means that there is only a finite number of possible values for the dependent value of a signal. 
This type of signal is called a \emph{\index{quantized}{quantized}} signal. We will not discuss quantization in this course.


\section{Systems}
%\newcommand{\sign}{\text{sign}}

\newthought{A signal processing \index{system}{system}} can be represented graphically as a block diagram, which describes signals going into a system, and signals going out of a system. 
An example is shown in Figure \ref{simple_sps}

\begin{figure}
\tikzstyle{int}=[draw]
%\tikzstyle{init} = [pin edge={to-,thin,black}]
\tikzstyle{init} = []
% [pin edge={to-,thin,black}]

\begin{center}
  \begin{tikzpicture}[node distance=4cm,auto,line width=0.1mm,>=triangle 45]
    \node [int] (a) {System $\spop$};
    \node (b) [left of=a,node distance=4cm, coordinate] {a};
    %\node [int, pin={[init]above:$p_0$}] (c) [right of=a] {$\frac{1}{s}$};
    \node [coordinate] (end) [right of=a, node distance=4cm]{};
    \path[->] (b) edge node {signal in $x(t)$} (a);
    %\path[->] (a) edge node {$v$} (c);
    \draw[->] (a) edge node {signal out $y(t)$} (end) ;
\end{tikzpicture}
\end{center}
\label{simple_sps}
\caption{A simple signal processing system block diagram.}
\end{figure}
A graphical representation is useful for understanding a signal processing system, especially if it is a complicated one, which includes many systems and signals.

The following block diagram is a real-world example of the block diagram of the EISCAT Svalbard radar transmitter subsystem. 
\begin{figure}
\begin{center}
\includegraphics[width=\textwidth]{ch04/figures/wannberg97.jpg}
\end{center}
\caption{An example of a more complicated block diagram. From: Wannberg et.al., 1997.}
\label{fig:esr}
\end{figure}

The mathematical description of a system (what goes on in the box) is a general transformation of a signal. 
The mathematical notation for a continuous-time system in general is\footnote{Here $\spop$ is a mathematical description of how the input signal is modified by the signal processing system. This is also called an
\index{operator}{operator}}:
\begin{equation}
\boxed{
y(t) = \spop[x(t)]
}
\end{equation}
and for discrete-time system:
\begin{equation}
\boxed{
y[n] = \spopb\{x[n]\}
}
\end{equation}
An example of a system could be a function that delays the input
signal by a delay $\tau$:
\begin{equation}
y(t) = x(t-\tau).
\end{equation}
Another example is a linear amplifier, which multiplies the amplitude
of the signal with a constant $\alpha \in \mathbb{C}$:
\begin{equation}
y(t) = \alpha x(t).
\end{equation}
You will often encounter both of these types of systems.

\newthought{For example, we can use these two basic systems to make a simplified model of a radar}. Imagine that you have a radar. This radar sends out a waveform that is described by a waveform $x(t)$. Let's say that the
time it takes an electromagnetic wave to travel from the radar transmitter to a point-like radar target and back is $\tau$ seconds. The received radar echo will be a delayed version of the transmitted signal, which is delayed by $\tau$. In addition to this, the signal will be scaled, as a radar echo is typically much weaker than the transmitted signal.

Therefore, a very simple system that describes a radar echo is the following:
\begin{equation}
y(t) = \alpha x(t-\tau)
\end{equation}
where $y(t)$ is the received radar echo signal.

We can further expand this concept to write the equation that describes a radar echo for a situation where there can be a continuous radar echo at time delays between $0$ and $T$:
\begin{equation}
y(t) = \int_0^T \alpha(\tau) x(t-\tau) d\tau.
\end{equation}
In this case, $\alpha(\tau)$ describes the radar echo amplitude as a function of propagation delay. This type of equation is called a \emph{\index{convolution}convolution} equation. 
You will encounter this type of equation very often in signal processing, and find that the Fourier transform is a very useful tool when dealing with convolution equations.

\newthought{Differential operators can also be seen as
systems}. Consider the first time-derivative of the continuous-time signal $x(t)$:
\begin{equation}
y(t) = \frac{d}{d t}x(t).
\end{equation}
The system definition is the time derivative operator $\frac{d}{dt}$.

\newthought{Systems are classified based on their mathematical properties}. The most important two are: \emph{\index{linearity}{linearity}} and \emph{\index{time-invariance}{time-invariance}}. A
system which is both linear and time-invariant is called 
\index{linear time-invariant}{linear time-invariant} 
(LTI)\index{LTI}. Such systems have beneficial mathematical properties that make analysis and design of such systems very straightforward. 
We'll later on prove that LTI systems are fully characterized by something known as an impulse response. This is an important concept in signal processing.

\section{Linear system}
\index{Linear system}

\begin{marginfigure}[-3cm]
\tikzstyle{int}=[draw, minimum size=2em]
\tikzstyle{init} = [pin edge={to-,thin,black}]

\begin{center}
% Definition of blocks:
\tikzset{%
  block/.style    = {draw, thick, rectangle, minimum height = 3em,
    minimum width = 3em},
  sum/.style      = {draw, circle, node distance = 2cm}, % Adder
  input/.style    = {coordinate}, % Input
  output/.style   = {coordinate} % Output
}
% Defining string as labels of certain blocks.
\newcommand{\suma}{\Large$+$}
\newcommand{\inte}{$\displaystyle \int$}
\newcommand{\derv}{\huge$\frac{d}{dt}$}
\newcommand{\mula}{\Large$\cross$}

\begin{tikzpicture}[auto, line width=0.1mm, node distance=2cm, >=triangle 45]
% make an op block
\draw 
   node at (0,0) [block, name=op0] {$\spop$};
% make a sum block
\draw 
   node at (-1.5,0) [sum, name=suma2] {\suma};
% output node
\draw 
   node at (1.5,0) [name=out0] {$y_1(t)$};

% input node 1
\draw 
   node at (-3,1.5) [name=in0] {$x_1(t)$};

% input node 2
\draw 
   node at (-3,-1.5) [name=in1] {$x_2(t)$};

% make a mult block
\draw 
   node at (-1.5,1.5) [sum, name=mul0] {\mula};
% make a mult block
\draw 
   node at (-1.5,-1.5) [sum, name=mul1] {\mula};

% output node
\draw 
   node at (0,1.5) [name=c0] {$c_2$};

% output node
\draw 
   node at (0,-1.5) [name=c1] {$c_1$};
% join
\draw[->](suma2) -- node {}(op0);
\draw[->](mul0) -- node {}(suma2);
\draw[->](mul1) -- node {}(suma2);
\draw[->](op0) -- node {}(out0);
\draw[->](in0) -- node {}(mul0);
\draw[->](in1) -- node {}(mul1);
\draw[->](c0) -- node {}(mul0);
\draw[->](c1) -- node {}(mul1);
\end{tikzpicture}

\vspace{1em}
\begin{tikzpicture}[auto, line width=0.1mm, node distance=2cm, >=triangle 45]
% make an op block
\draw 
   node at (-1.5,1.5) [block, name=op0] {$\spop$};
\draw 
   node at (-1.5,-1.5) [block, name=op1] {$\spop$};
% make a sum block
\draw 
   node at (0,0) [sum, name=suma2] {\suma};
% make a mult block
\draw 
   node at (0,1.5) [sum, name=mul0] {\mula};
% make a mult block
\draw 
   node at (0,-1.5) [sum, name=mul1] {\mula};
% output node
\draw 
   node at (1.5,0) [name=out0] {$y_2(t)$};
% output node
\draw 
   node at (1.5,1.5) [name=c0] {$c_2$};
% output node
\draw 
   node at (1.5,-1.5) [name=c1] {$c_1$};
% input node 1
\draw 
   node at (-3,1.5) [name=in0] {$x_1(t)$};
% input node 2
\draw 
   node at (-3,-1.5) [name=in1] {$x_2(t)$};
% join
\draw[->](suma2) -- node {}(out0);

\draw[->](in0) -- node {}(op0);
\draw[->](in1) -- node {}(op1);

\draw[->](op0) -- node {}(mul0);
\draw[->](op1) -- node {}(mul1);
\draw[->](mul0) -- node {}(suma2);
\draw[->](mul1) -- node {}(suma2);
\draw[->](c0) -- node {}(mul0);
\draw[->](c1) -- node {}(mul1);

\end{tikzpicture}
\end{center}
\label{fig:linearity_block}
\caption{In order for the system specified by $\spop$ to be linear, $y_1(t) = y_2(t)$ must be satisfied.}
\end{marginfigure}

A system is linear, if a linear combination of inputs fed into the system yields the same as the linear combination of outputs:
\begin{equation}
\boxed{
\spop[c_1 x_1(t) + c_2 x_2(t)] = c_1 \spop[x_1(t)] + c_2 \spop[ x_2(t) ]}
\label{def:linear_sys}
\end{equation}
for arbitrary constants $c_1 \in \mathbb{C}$ and $c_2 \in \mathbb{C}$ and arbitrary input signals $x_1(t)$ and $x_2(t)$. This property is highly useful, and appears throughout signal processing.

\newthought{An example of a linear system} is a system that scales the input signal by a constant factor $\alpha$:
\begin{equation}
  y(t) = \alpha x(t).
\end{equation}
If $|\alpha| >1$, the signal is amplified. This type of system would typically be called an amplifier. If $0<|\alpha|<1$, the system would be called an attenuator, as the output signal amplitude would be attenuated.

It is quite easy to determine that the test for linearity is passed for this system:
\begin{equation}
\alpha [c_1 x_1(t) + c_2 x_2(t)] = c_1 [\alpha x_1(t)] + c_2 [\alpha x_2(t)].
\end{equation}

\newthought{An example of a non-linear system} is the following system, which obtains the absolute value of $x(t)$:
\begin{equation}
y(t) = |x(t)|
\end{equation}
It is quite clear that this does not pass the test for linearity:
\begin{equation}
|c_1 x_1(t) + c_2 x_2(t)| \ne c_1 |x_1(t)| + c_2 |x_2(t)|,
\end{equation}
for all possible values $c_1$, $c_2$, $x_1(t)$, and $x_2(t)$.
\begin{marginfigure}
\tikzstyle{int}=[draw, minimum size=2em]
\tikzstyle{init} = [pin edge={to-,thin,black}]

\begin{center}

% Definition of blocks:
\tikzset{%
  block/.style    = {draw, thick, rectangle, minimum height = 3em,
    minimum width = 3em},
  sum/.style      = {draw, circle, node distance = 2cm}, % Adder
  input/.style    = {coordinate}, % Input
  output/.style   = {coordinate} % Output
}
% Defining string as labels of certain blocks.
\newcommand{\suma}{\Large$+$}
\newcommand{\inte}{$\displaystyle \int$}
\newcommand{\derv}{\huge$\frac{d}{dt}$}
\newcommand{\mula}{\Large$\cross$}

\begin{tikzpicture}[auto, line width=0.1mm, node distance=2cm, >=triangle 45]
% make an op block
\draw  
node at (-1.0,3.0) [name=in0] {$x(t)$};

% make an op block
\draw  
node at (-1.0,1.5) [block, name=op0] {$\spop$};

\draw  
node at (-1.0,0) [block, name=delay0] {$\mathcal{D}\{\cdot\}$};

% output node
\draw 
 node at (-1.0,-1.5) [name=out0] {$y_2(t)$};

% make an op block
\draw  
node at (1.0,3.0) [name=in1] {$x(t)$};

% make an op block
\draw  
node at (1.0,0.0) [block, name=op1] {$\spop$};

\draw  
node at (1.0,1.5) [block, name=delay1] {$\mathcal{D}\{\cdot\}$};

% output node
\draw 
 node at (1.0,-1.5) [name=out1] {$y_1(t)$};

% join
\draw[->](in0) -- node {}(op0);
\draw[->](op0) -- node {}(delay0);
\draw[->](delay0) -- node {}(out0);


% join
\draw[->](in1) -- node {}(delay1);
\draw[->](delay1) -- node {}(op1);
\draw[->](op1) -- node {}(out1);


\end{tikzpicture}
\end{center}
\caption{In order for the system specified by $\spop$ to be time-invariant, $y_1(t) = y_2(t)$ must be satisfied. \label{fig:time_inv_block}}
\end{marginfigure}

\section{Time-invariant system}
\index{Time-invariant}

Let us first define a delay system $\mathcal{D}\{x(t)\} = x(t-\tau)$. Let us assume that the output of a system is $y(t) = \spop[x(t)]$. This system is time-invariant if:
\begin{equation}
\boxed{
\mathcal{T}\{\mathcal{D}\{x(t)\}\} = \mathcal{D}\{\mathcal{T}\{x(t)\}\}
\label{def:timeinv}
}
\end{equation}
In other words, it does not matter if the signal is delayed before or after the system $\mathcal{T}\{\cdot\}$. To check time-invariance, we need to verify Equation \ref{def:timeinv}. 

The test for time-invariance is illustrated as a block diagram in Figure \ref{fig:time_inv_block}. 
If you want, you could think of a time-invariant system as a system that satisfies:
$$[\mathcal{T},\mathcal{D}]x(t)=0,$$
where $[\mathcal{T},\mathcal{D}]=\mathcal{T}\mathcal{D}-\mathcal{D}\mathcal{T}$ is the commutator of two operators. 
Thus, a time-invariant system $\mathcal{T}$ is a system that commutes with the time-delay operator $[\mathcal{T},\mathcal{D}]=0$. 

\newthought{An example of a time-invariant system} is the system that returns the absolute value of the input signal $y(t)=|x(t)|$. We can see this by evaluating:
\begin{align}
\mathcal{T}\{\mathcal{D}\{x(t)\}\} &= \mathcal{T}\{x(t-\tau)\}=|x(t-\tau)|\\
\mathcal{D}\{\mathcal{T}\{x(t)\}\} &= \mathcal{D}\{|x(t)|\}=|x(t-\tau)|
\end{align}
Time-invariance holds as the outputs are equal. It doesn't matter if a time-delay is applied to the signal before or after obtaining the absolute value of the input signal.

\newthought{An example of a case that is not time-invariant} is $y(t) = \mathcal{T}\{x(t)\} = t + x(t)$. We can immediately see that the system directly depends on $t$, not only through, the input. 
The formal test also shows that time-invariance is not met:
\begin{align}
\mathcal{T}\{\mathcal{D}\{x(t)\}\} &= \mathcal{T}\{x(t-\tau)\}=t+x(t-\tau) \\
\mathcal{D}\{\mathcal{T}\{x(t)\}\} &= \mathcal{D}\{t + x(t)\}=(t-\tau)+x(t-\tau)
\end{align}
In this case, $\mathcal{T}\{\mathcal{D}\{x(t)\}\}\neq\mathcal{D}\{\mathcal{T}\{x(t)\}\}$ so the system is not time-invariant.

\section{Example: Overdriven amplifier}
\begin{marginfigure}
\begin{tikzpicture}
	\begin{axis}[
		xmin=-3, xmax=3,
		height=7cm,
		width=7cm,
		axis x line=center,
		axis y line=center,
                ylabel=$y(t)$,
		xlabel=$x(t)$
	]
		
	\addplot[mark=none,color=blue] {2*x};
	\addlegendentry{$y(t)$}

	\addplot[mark=none,color=red] coordinates {
		(-100,-3)
		(-1.5,-3)
		(1.5,3)
		(100,3)
	};
	\addlegendentry{$y_d(t)$}
	\end{axis}
\end{tikzpicture}
\caption{The system function of a linear amplifier $y(t)$ and a clipping amplifier system $y_d(t)$.}
\label{dist_effect}
\end{marginfigure}

A simple model of a distorted guitar amplifier system would be the
following clipping amplifier system, which is specified as follows:
\begin{equation}
y_d(t) = \left\{
  \begin{array}{rcr}
    -\beta & \mathrm{when} & \alpha x(t)<-\beta \\
    \alpha x(t) & \mathrm{when} & |\alpha x(t)| \le \beta \\
    \beta & \mathrm{when} & \alpha x(t)>\beta 
\end{array}
\right.
\label{clipamp}
\end{equation}
This is a very approximative model of an overdriven guitar
amplifier. This type of system is often encountered in guitar music from the 50s and onwards. I'm sure that once you later implement this in practice, you'll recognize the sound that this system makes.

What does this system do? It amplifies the signal, but only up to a certain point. Beyond a certain absolute value of the input, the output maintains a constant positive or negative value.  This type of behavior is also sometimes called clipping, and the effect is
also sometimes called distortion, as the input signal amplitude is not linearly scaled, but it is rather distorted.

\begin{marginfigure}
  \begin{center}
    \includegraphics[width=\textwidth]{ch04/figures/ampvid.jpg}
  \end{center}
  \caption{A video discussing an overdriven guitar amplifier can be found here: \url{https://youtu.be/I30Mn_-yYF8}.}
\end{marginfigure}
Many real world amplifier systems have this type of saturation behavior. What this often means in practice is that the system is linear when the input absolute amplitude is less than some critical value, but beyond this, linearity no longer holds.

Figure \ref{dist_effect} illustrates what $y_d(t)$ would look like as a function of $x(t)$. It is compared with a normal amplifier system $y(t)=\alpha x(t)$, which does not clip.

\if 0
\begin{center}
\includegraphics[width=0.68\textwidth]{ch01_guitar_amp/clipamp.png}
\end{center}
\fi
