\newpage
\section{Exercises: Fourier Transform}
\begin{enumerate}
  % Exercise 1
  \item Find the Fourier transforms of these signals:
        \begin{enumerate}[a)]
          % Exercise 1a)
          \item $x(t)=\delta(t+2)+\delta(t)+\delta(t-2)$ \\Hint: linearity and time shift properties.
          % Exercise 1b)
          \item $x(t)=\frac{\sin(100\pi [t-2])}{\pi (t-2)}$ \\Hint: time shift property.
          % Exercise 1c)
          \item $x(t)=e^{-t}u(t) - e^{-t} u(t-4)$ \\Hint: look at the derivation of 
          the Fourier transform of $e^{-\beta t} u(t)$.
        \end{enumerate}
        Note that there are more than one way to solve the Fourier transforms!

  % Exercise 2
  \item Find the inverse Fourier transform for the following:
        \begin{enumerate}[a)]
          % Exercise 2a)
          \item $\hat{x}(\omega)= 2 + \cos(\omega)$
          % Exercise 2b)
          \item $\hat{x}(\omega) = \frac{1}{1+i\omega} - \frac{1}{2+i\omega}$
          % Exercise 2c)
          \item $\hat{x}(\omega) = i \delta(\omega - 100\pi) - i\delta(\omega + 100\pi)$.
        \end{enumerate}

  % Exercise 3
  \item Use known Fourier transform pairs together with the properties of the Fourier 
        transform to complete the following Fourier transform pairs:
        \begin{enumerate}[a)]
          % Exercise 3a)
          \item $x(t) = u(t+3)u(3-t) \xleftrightarrow{\mathcal{F}} \hat{x} = ?$
          % Exercise 3b)
          \item $x(t) = \sin(4\pi t) \sin(50 \pi t)  \xleftrightarrow{\mathcal{F}} \hat{x} = ?$
          % Exercise 3c)
          \item $x(t) = \frac{\sin(4\pi t)}{\pi t} \sin(50 \pi t)  \xleftrightarrow{\mathcal{F}} \hat{x} = ?$
          % Exercise 3d)
          \item $x(t) = ?  \xleftrightarrow{\mathcal{F}} \hat{x} = \cos^2(\omega)$
        \end{enumerate}

  % Exercise 4
  \item Let $\hat{x}(\omega)$ be the Fourier transform of $x(t)$. Prove that if $y(t)=x(t-\tau)$, 
  then $\hat{y}(\omega)=e^{-i\omega\tau}\hat{x}(\omega)$.




\end{enumerate}
