\newpage
\section{Suggested solutions: Ideal and Tapered Filters}

\begin{marginfigure}
\begin{center}
        \begin{tikzpicture}
        \begin{axis}[width=7cm,height=6cm,ymin=0,xmin=-3.0,ymax=1.5,xmax=3.0,  yticklabels={,,},
        xtick={-2.5,-1.5,-0.5,0.5,1.5,2.5},
        xticklabels={$-\pi$,$-\omega_1$,$-\omega_0$,$\omega_0$,$\omega_1$,$\pi$},
        ylabel=$\He$,
    xlabel=$\hat{\omega}$, axis lines = center]

\addplot[mark=none,color=blue] coordinates {
		(-2.5,1)
		(-1.5,1)
		(-1.5,0)
		(-0.5,0)
        (-0.5,1)
	    (0.5,1)
        (0.5,0)
        (1.5,0)
        (1.5,1)
        (2.5,1)
};
        \end{axis}
        \end{tikzpicture}
\end{center}
\caption{The frequency response of an ideal stop-pass filter.}
\label{fig:ideal_bs_fr}
\end{marginfigure}

\begin{enumerate}
\item Let the ideal band-stop filter be given as:
\begin{equation*}
\mathcal{H}_{\mathrm{BS}}(\hat{\omega}) = \left\{ \begin{array}{cc}
0 & \hat{\omega}_0 < |\hat{\omega}| < \hat{\omega}_1 \\
1 & \mathrm{otherwise}
\end{array}\right.\,\,.
\end{equation*}
The impulse response can then be computed using the inverse DTFT as follows (see Figure \ref{fig:ideal_bs_fr}):
\begin{align*}
    h[n]&=\frac{1}{2\pi}\int_{-\pi}^{\pi}\mathcal{H}_{\mathrm{BS}}(\hat{\omega})e^{i\hat{\omega}n}d\hat{\omega}, \\
    &=\frac{1}{2\pi}\int_{-\pi}^{-\hat{\omega}_{1}}e^{i\hat{\omega}n}d\hat{\omega} + \frac{1}{2\pi}\int_{-\hat{\omega}_{0}}^{\hat{\omega}_{0}}e^{i\hat{\omega}n}d\hat{\omega} + \frac{1}{2\pi}\int_{\hat{\omega}_{1}}^{\pi}e^{i\hat{\omega}n}d\hat{\omega}, \\
    &=\frac{1}{2\pi}\left[\frac{1}{in}e^{i\hat{\omega}n}\right]_{-\pi}^{-\hat{\omega}_{1}} + \frac{1}{2\pi}\left[\frac{1}{in}e^{i\hat{\omega}n}\right]_{-\hat{\omega}_{0}}^{\hat{\omega}_{0}} + \frac{1}{2\pi}\left[\frac{1}{in}e^{i\hat{\omega}n}\right]_{\hat{\omega}_{1}}^{\pi}, \\
    &=\frac{1}{2in\pi}\left(e^{-i\hat{\omega}_{1}n}-e^{-i\pi n}\right) + \frac{1}{2in\pi}\left(e^{i\hat{\omega}_{0}n}-e^{-i\hat{\omega}_{0}n}\right) + \frac{1}{2in\pi}\left(e^{i\pi n}-e^{i\hat{\omega}_{1}n}\right), \\
    &=\frac{1}{n\pi}\sin(\pi n) + \frac{1}{n\pi}\sin(\hat{\omega}_{0}n) - \frac{1}{n\pi}\sin(\hat{\omega}_{1}n), \\
    &=\delta[n] + \frac{\sin(\hat{\omega}_{0}n)}{n\pi} - \frac{\sin(\hat{\omega}_{1}n)}{n\pi}.
\end{align*}
As expected, the ideal stop-pass is just the opposite of an ideal band-pass filter. 


\item Continuing with the filter in the previous exercise. 

\begin{enumerate}[a)]
\item The impulse response of the band-stop filter was shown in the previous exercise to be:
$$h[n]=\delta[n] + \frac{\sin(\hat{\omega}_{0}n)}{n\pi} - \frac{\sin(\hat{\omega}_{1}n)}{n\pi}.$$
A tapered version of this filter is then found by:
$$h_{w}[n]=w(n)h[n - N/2],$$
where $N$ is the length of the filter and the window function. Putting all of this together, we obtain: 
$$h_{w}[n]=\delta[n-N/2]w[n] +  \frac{\sin(\hat{\omega}_{0}(n-N/2))}{(n-N/2)\pi}w[n] - \frac{\sin(\hat{\omega}_{1}(n-N/2))}{(n-N/2)\pi}w[n].$$

\item An implementation of the band-stop filter is shown in Listing \ref{solex13_2}.

\lstinputlisting[language=Python,caption=Band-stop filter,label=solex13_2]{ch13/code/ex13_2.py}
\begin{marginfigure}
\includegraphics[width=7.0cm,height=6.8cm]{ch13/figures/impulse_response.png}
\caption{Impulse response for the band-stop filter}
\label{impulse_stop}
\end{marginfigure}

In Figure \ref{impulse_stop} a plot of the impulse response for the band-stop filter is shown, both in time and frequency domain. 
Figure \ref{freq_spec_filter} shows that the frequency components corresponding to the band between $f_{0}$ and $f_{1}$ have been
reduced in power. This is what one would except from a band-stop filter.

\begin{marginfigure}
\includegraphics[width=7.5cm,height=7.0cm]{ch13/figures/freq_spec_filter.png}
\caption{Frequency spectrum of the filtered signal}
\label{freq_spec_filter}
\end{marginfigure}
    
\item Using $f_{0}=190$ Hz and $f_{1}=$ Hz to make sure we include both frequencies in the band to filter out the lower frequencies. 
When the audio is played the audio contains only a high frequency pitch, as hoped. 

\end{enumerate}
\end{enumerate}