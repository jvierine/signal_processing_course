\chapter{Programming Assignment 3}

\begin{marginfigure}
\begin{center}
\includegraphics[width=\textwidth]{Assignments/figures/nsf_ligo.jpg}
\end{center}
\caption{Artist's depiction of gravitational waves created by a merger of two neutron stars. Credits: US National Science Foundation.}
\end{marginfigure}

In 2017, the Nobel Prize in physics was awarded to Rainer Weiss, Barry
Barish and Kip Thorne for the discovery of gravitational waves. Only
two years earlier, on September 14, $^{\mathrm{th}}$ 2015, 09:50:45 UTC,
the two Laser Interferometer Gravitational-Wave Observatory (LIGO)
instruments detected a gravitational wave for the first time in
history. The existence of gravitational waves had been predicted by
Einstein's general theory of relativity, but never before detected in
situ.

The first gravitational wave event detected by LIGO is thought to be
created from a collision of two black holes, which sends out a
localized chirp like pulse in the space-time. LIGO utilizes two
detectors, which are spaced 3000 km apart. These detectors measure
\emph{strain} ($\Delta L/L$) as a function of time. Here $\Delta L$ is
the variation in length and $L$ is the total length in which the
variation is measured. In other words, strain is the normalized
variation in length $L$ of the interferometer line due to
gravitational-wave. Figure \ref{fig:ligo_nobel_diag} provides a high
level overview of the measurement.

LIGO uses two geographically separated stations to measure gravitational
waves: Hanford (H$_1$), and Livingston (L$_1$). The gravitational wave
propagates at the speed of light $c\approx 3\cdot 10^8$
$\mathrm{m}/\mathrm{s}$. If the same signal is detected at two
different places with a time difference less than or equal to the
speed of light propagation time between the sensors, then this
provides more confidence that the event is in fact real, and not
caused by for example local seismic activity. A third sensor would
allow determining the direction of arrival based on time of arrival.

\begin{marginfigure}
\begin{center}
\includegraphics[width=\textwidth]{Assignments/figures/hanliv.jpg}
\end{center}
\caption{The Hanford and Livingston interferometers. Credits: LIGO.}
\end{marginfigure}

The LIGO data is severely corrupted with instrumental noise. This
noise is much larger in amplitude than the gravitational wave
signal. However, this noise is very narrowband in nature, and it can
be filtered out using relatively basic signal processing techniques
without affecting the relatively broad band gravitational signal very
much. Such filtering is routinely used by LIGO to improve the
sensitivity of the instrument.

Verification of scientific results is an important part of
science. UiT does not have the resources for building a giant
interferometer, and this course really isn't about solving Einstein's
field equations, so we won't be able to reproduce all the
results. However, we can verify the signal processing part. In this
programming assignment, your task is to independently develop signal
processing software to verify that you can also find the gravitational
wave signature in the LIGO measurements.

\section{Instructions}

We expect you to complete the listed signal processing tasks. The
submission form is a written report, which answers the questions given
in each part of this assignment. The report should include the code
with comments that indicate what each part of the program does.

% In the first part, you have already completed sections 3-7. For this
% second part of the assignment, you need to complete sections
% 8-13. You may choose to continue your previous report, or to write a
% new report that just covers sections 8-13.

You will only need to know about signal processing concepts taught in
FYS-2006. You do not need to know anything about gravitational waves
or the LIGO instrument in order to complete the assignment. The
lecture notes for the course contain several helpful signal processing
examples. Take a look at lecture notes for Week 43 on \emph{spectral
analysis}, and \emph{arbitrary frequency response filters}.


\section{Instruction for reading the data}

To begin, download the data files and a simple program that
demonstrates reading these files from this location:
\url{https://bit.ly/3bIcRlB}. These data files contain real
measurements from LIGO starting at 2015-09-14T09:50:30 UTC. After
downloading the files, the next step is to read the strain signal from
the data files. In Python, this can be done using the h5py module\sidenote{Make sure you have the h5py module installed on your computer!}.
\begin{verbatim}
import h5py
h=h5py.File("file.hdf5","r")
data=h["strain/Strain"][()]
\end{verbatim}
You will need to read two data vectors. One for the Livingston station
(L$_1$) and one for the Hanford station (H$_1$). We will use the symbol $x_H[n]$
of the Hanford signal and $x_L[n]$ for the Livingston signal.

\section{1. Data}
The sample rate of both of the signals is $f_s=4096$ Hz. The samples
in both signals are synchronized in time, i.e., sample $n$ in signal
$x_H[n]$ and $x_L[n]$ occur at the same time.

\begin{enumerate}[a)]

\item Write code to read the Hanford and Livingston signals from the data file.
  
\item How many samples are in each of the signals: $x_H[n]$ and $x_L[n]$?
  
\item How many seconds of signal does each of the two data vectors $x_H[n]$ and $x_L[n]$ represent?
  
\item The gravitational wave signal is a one dimensional real valued
  signal that is spectrally confined to the frequency range $[-300,
    300]$ Hz. Is the signal sampled at a sufficiently high sample rate
  to retain all the gravitational wave information?
  
\item What is the sample spacing in units of seconds?
\end{enumerate}

\section{2. Plotting the data}

In order to see what the signals look like, you will need to plot the data.
\begin{enumerate}[a)]
\item Plot the signals $x_H[n]$ and $x_L[n]$, with time in seconds on
  the horizontal axis and strain on the vertical axis. Assume that
  time at the beginning of the signal array starts at $0$ seconds. Label
  the axes of your plot. Use separate plots for $x_H[n]$ and $x_L[n]$
  signals. Hint: you can use the \verb|plt.plot(t,signal)| command found in Matplotlib. Use an array \verb|t| to denote the seconds of each sample of the array \verb|signal|.
  
\item What are the minimum, maximum, and mean values of the $x_H[n]$ and $x_L[n]$ signals?
  
\end{enumerate}

\section{3. Selecting a tapered window function}

You will need to apply a discrete Fourier transform to analyze the
spectral content of the signal. You will need to select a suitable
tapered window function in order to obtain a good rejection of out of
band signals. In this task, we'll use a synthetic narrowband signal to
compare the performance of a discrete Fourier transform (DFT) based spectral
analysis using a tapered window to spectral analysis done without a
tapered window.

In order to calculate the magnitude spectrum of a signal $x[n]$ of length $N$, you need to evaluate a DFT on the signal:
\begin{equation}
\hat{x}[k] = \sum_{n=0}^{N-1} x[n] e^{-i\frac{2\pi}{N}kn}.
\label{one}
\end{equation}
and the windowed signal:
\begin{equation}
\hat{x}_w[k] = \sum_{n=0}^{N-1} w[n]x[n] e^{-i\frac{2\pi}{N}kn}.
\end{equation}
Hint: Use the \verb|fft| function to evaluate the
DFT. This function is available in Python as \verb|numpy.fft.fft|.


\begin{enumerate}[a)]
\item Chose a tapered window function and implement it. Hint: you can use window functions available in the \verb|scipy.signal| module.

\item In order to get an idea of how your window function behaves,
  apply it to a sinusoidal signal
\begin{equation}
  x[n]=\cos(2\pi f  n/f_s)
\end{equation}
with frequency $f=31.5$ Hz, which is sampled at $f_s=4096$
Hz. Sample signal at $n\in[0,1,2,\cdots,4095]$. Make a plot of the
signal $x[n]$ and the windowed signal $w[n]x[n]$. Use a window of
the same length as your signal $N=4096$. Hint: You can use \verb|numpy.arange(N)| to create a sequence of integers between $0$ and $N-1$.

\item The FFT algorithm will evaluate $\hat{x}[k]$ at integer values of $k$ between $0$ and $N-1$. What frequencies $f_k$ in hertz do frequencies 
$\hat{\omega}_k = 2\pi k/N$ in radians per sample correspond to on the principal spectrum ($-f_s/2 < f_k < f_s/2$)? 

\item Which values of $k$ correspond to a frequency $f_k$ that is nearest to $31.5$ and $-31.5$ hertz?

\item Estimate the power spectrum of the windowed signal $w[n]x[n]$ (with tapering) and the
  signal $x[n]$ (without tapering). Plot the power spectrum in decibel scale (power) for both.
  Use frequency in Hz on the horizontal axis and magnitude squared
  $10 \log_{10}(|\hat{x}[k]|^2)$ (decibels) on the vertical axis.  Plot both positive and negative frequencies. Hint: You can use \verb|numpy.fft.fftfreq| and \verb|numpy.fft.fftshift| to determine what frequency (in hertz) each FFT bin $k$ corresponds to, and to order the frequencies in an ascending order.

  \item Mark the locations of 31.5 and -31.5 Hz on the plot of the power spectrum.


\item The frequency corresponding to $\hat{\omega}=\pi$ radians per
  second or $f=2048$ Hz is non-zero for both $\hat{x}[k]$ and $\hat{x}_w[k]$. However, your test signal has a frequency of 31.5 Hz. Why is there a non-zero frequency component anywhere else than at 31.5 Hz when you analyze the signal?

\item How many decibels is the frequency response better with the
  tapered window function ($10\log_{10}(|\hat{x}_w[k]|^2)$) than without
  ($10\log_{10}(|\hat{x}[k]|^2)$) at frequency $\hat{\omega}=\pi$ (radians per
  second)?

\end{enumerate}

\section{4. Estimating the spectrum of the LIGO signal}

\begin{enumerate}[a)]
\item Calculate the power spectrum of the LIGO signals
  $|\hat{x}_L[k]|^2$ and $|\hat{x}_H[k]|^2$. Here $\hat{x}_L[k]$ is the
  windowed DFT of the Livingston signal $x_L[n]$ and $\hat{x}_H[k]$ is
  the windowed DFT of the Hanford signal $x_H[n]$. The windowed DFT is
  obtained using
\begin{equation}
\hat{x}[k] = \sum_{n=0}^{N-1} w[n]x[n]e^{-i\frac{2\pi}{N}kn}
\label{dfteq}
\end{equation}
Perform the DFT over the whole dataset, i.e., $N$ is the number of samples in the whole signal vector. Use the window function that you have chosen in the previous exercise, but make sure that use a window of length $N$, where $N$ is the LIGO data vector length. Hint: Use FFT.

\item Plot the results with frequency in Hz in the horizontal axis and
  power using decibel scale on the vertical axis. Make one
  plot for the Hanford and one plot for the Livingston. Plot only the positive frequencies.

\item On what frequencies are there strong spectral components in the
  Hanford and Livingston power spectra? Use Hz as the unit of
  frequency. Identify up to 12 frequency bands that contain narrowband interference. Label these regions on the plot of the power spectrum.

\end{enumerate}

\section{5. Whitening filter}

In order to remove instrumental noise, you will next implement a
whitening filter and apply it to the LIGO signal. A whitening filter
is a filter that modifies the amplitudes of each spectral component in
such a way that the magnitude spectrum of the filter output is constant-valued.

A whitening filter can be implemented as an FIR filter, which in frequency domain can be implemented as a multiplication:
\begin{equation}
\hat{y}[k] = \hat{h}[k]\hat{x}[k].
\end{equation}
Here $\hat{y}[k]$ is the DFT of the output of the filter, $\hat{h}[k]$ is the DFT of the whitening filter, and $\hat{x}[k]$ is the windowed DFT of the input signal $x[n]$. 

The purpose of a whitening filter is to filter the signal in such a way that the magnitude of the output signal is unity: $|\hat{y}[k]|=1$. This can be  obtained using a filter of the form:
\begin{equation}
\hat{h}[k]=\frac{1}{|\hat{x}[k]|}.
\label{wfilt}
\end{equation}
\begin{enumerate}[a)]

\item Show that $\hat{h}[k]$ as defined in equation \ref{wfilt} will filter signal $x[n]$ in such a way that $|\hat{y}[k]|=1$. 

%\item Will $\hat{h}[k]$ be the same for the Hanford and Livingston
%  signals $x_H[n]$ and $x_L[n]$? Why?

\item Implement a whitening filter in frequency domain $\hat{h}[k]$
  for the LIGO data. Implement a separate filter for the Hanford and
  Livingston signals. Use all the signal as input to the windowed
  DFT when calculating $\hat{x}[k]$. Do not filter the signal in
  smaller blocks!

\item Use an inverse discrete Fourier transform to transform the
  whitened signal $\hat{y}[k]$ into time-domain. Do this for both
  Livingston and Hanford signals separately (obtaining $y_L[n]$ and
  $y_H[n]$). 

\item Plot the whitened signals $y_H[n]$ and $y_L[n]$ for both Hanford
  and Livingston. Use x-axis for time in seconds
  ($t\in[0,t_{\mathrm{max}}]$) and y-axis for whitened strain
  $y[n]$. The gravitational wave signal is in the middle of the
  signal, between 16.2 and 16.5 seconds. If you've done everything
  correctly, you should see the gravitational wave signal. It looks
  like a chirp (see plot in Figure \ref{fig:ligo_result_plot}). However, because we are not
  done yet, the signal will look noisy.

\end{enumerate}

\section{6. Low-pass filtering}

The gravitational wave signal is at frequencies below 300 Hz. Design a simple running mean averaging low-pass filter
\begin{equation}
y[n] = \frac{1}{L} \sum_{k=0}^{L-1}x[n-k]
\end{equation}
that will attenuate spectral components with frequencies above 300 Hz. 
\begin{enumerate}[a)]
\item Find an integer value of $L$ such that the filter will reduce the power of frequency components at $f=300$ Hz by approximately -6 dB compared to the filter output for a $f=0$ Hz signal.
  
\item Plot the power spectral response of the filter in dB scale ($10\log_{10}|\He|^2$), where $\He$ is the discrete-time Fourier transform of the FIR filter coefficients of the averaging filter. Use frequency on the horizontal axis in Hz. Label the -6 dB point in frequency and in power spectral response, verifying that the -6 dB point is close to 300 Hz.

\item What is the time delay $\tau$ to the signal introduced by the filter, in seconds?

\item Apply the running mean average low-pass filter on the whitened
  Hanford and Livingston signals ($y_H[n]$ and $y_L[n]$).

\item Undo the effects of the filter time-delay by shifting the signal in time. You can do this by adjusting the time variable $t'=t-\tau$, instead of filtering the signal.

\item Plot the low pass filtered whitened Hanford and Livingston signals. You should see the gravitational wave signal more clearly now. Use the horizontal axis for time and the vertical axis for the signal amplitude. Plot only the time interval between 16.1 and 16.6 seconds where the gravitational wave signal is located. Compare your plot with Figure \ref{fig:ligo_result_plot}. Your plot should be similar, but not necessarily exactly the same, as your filter is not exactly the same as the one used for Figure \ref{fig:ligo_result_plot}.


%\item Why is the gravitational wave signal more clearly visible now?

\end{enumerate}

\begin{figure}
\begin{center}
\includegraphics[width=\textwidth]{Assignments/figures/dc_fg.png}
\end{center}
\caption{The figure is from: B. P. Abbott et al. (LIGO Scientific Collaboration and Virgo Collaboration)
  Phys. Rev. Lett. 116, 061102 – Published 11 February 2016}
\label{fig:ligo_result_plot}
\end{figure}


\section{7. Time delay}

Gravitational waves are expected to propagate at the speed of
light. The Hanford and Livingston detectors are separated by about
3000 km. A gravitational wave will propagate this distance in about 10
ms.

The gravitational wave angle of arrival is not known, but the relative
time delay between the signal detected at Hanford and Livingston is
expected to be $-10 < \tau < 10$ ms, if it is moving at the speed of
light in vacuum.
\begin{enumerate}[a)]
  
\item  Determine the time separation between the two signals \emph{by plotting the magnitudes} of the filtered gravitational wave signals ($|y_L[n]|$ and $|y_H[n+n_0]|$) with different delays $n_0$ on the same plot. Try different values of $n_0$ until the signals are approximately aligned in time. The magnitude of the signal is used to avoid phase-time ambiguities. Hint: Use magnitude, not amplitude. Magnitude can be obtained using \verb|numpy.abs|.
  
\item What value do you obtain for the sample delay $n_0$?
  
\item What time delay $\tau$ in seconds does the sample delay $n_0$ correspond to?
  
\item Is the time delay $\tau$ in agreement with gravitational-wave propagation speed (i.e., that $-10 < \tau < 10$ ms)?
  
\end{enumerate}

\section{8. Dynamic spectrum}

\begin{marginfigure}
\begin{center}
\includegraphics[width=\textwidth]{Assignments/figures/dynspec.png}
\end{center}
\caption{A dynamic spectrum plot of the gravitational wave signal.}
\label{fig:dynspec_ligo_ex}
\end{marginfigure}

Study the time-frequency behavior of the gravitational wave signal.
\begin{enumerate}[a)]
  
\item Calculate the dynamic spectrum (spectrogram) of the low-pass
  filtered whitened signal $|\hat{x}[t,k]|$, where $t$ is time and $k$
  is frequency.

\item Plot the dynamic power spectrum $|\hat{x}[t,k]|^2$ in dB scale. Perform this in such a way that you can see the time-frequency response behavior of the gravitational wave signal clearly. Plot your result at between $t=15.5$ s and $t=17$ s. Hint: You will need to use zero-padding, a tapered window, and overlapping windows. An example of the time-frequency domain behavior of the gravitational wave signal is shown in Figure \ref{fig:dynspec_ligo_ex}.

  
\end{enumerate}


\section{9. Extra task}
For earning an extra point, improve any part of the signal processing
in a way that you see fit. Document your improvements. You can, e.g.,
try to create a filter that removes only the strong frequency
components from the signal, or you can use a better low-pass filter.

\begin{figure}
\begin{center}
\includegraphics[width=\textwidth]{Assignments/figures/ligo_nobel.jpg}
\end{center}
\caption{The gravitational wave measurement using a Michelson-Morley interferometer. Credits: Johan Jarnestad, The Royal Swedish Academy of Sciences.}
\label{fig:ligo_nobel_diag}
\end{figure}

