\chapter{Programming Assignment 2}

\begin{marginfigure}[5cm]
  \begin{center}
    \includegraphics[width=\textwidth]{Assignments/figures/bat.jpg}
  \end{center}
  \caption{Bats use chirp-like ultrasound signals to sense their surroundings. Image: David Dennis.}
  \label{fig:bat_image}
\end{marginfigure}

This programming assignment deals with deconvolution of transmit
signals for an \emph{acoustic sounder} or a \emph{sonar}. A sonar is
a device that measures the amplitude of sound waves that are scattered
from objects at different propagation distances between the acoustic
wave transmitter and receiver. These devices are used, e.g., in cars to
assist with parking, in ships to measure the water depth, and for
non-invasive medical examinations. Sonars are also closely related
with radars, as they share many of the same signal processing
concepts. The main difference is that radars use electromagnetic waves
instead of acoustic waves.

In order to implement a simple sonar, I have used a loudspeaker and a
microphone connected to a sound card to transmit and receive acoustic
signals. The sampling rate of the audio recording is $f_s=44.1\cdot
10^3$ hertz, which results in a sample-spacing of $T_s=2.2676\cdot
10^{-5}$ s. The loudspeaker and the microphone are spaced apart by
approximately 2 cm.
 
If we ignore measurement errors, the acoustic signal can be modeled using
a discrete-time convolution equation:
\begin{equation}
  m[n] = \sum_{r=0}^{R-1} h[r]x[n-r] = h[n] * x[n]
  \label{eq:sonar_eq_pa}
\end{equation}
In this equation $m[n]$ is the signal measured with a microphone,
$x[n]$ is the signal transmitted by the loudspeaker, and $h[n]$ is
the amplitude of the scattered signal as a function of range. The
signal $h[n]$ is also the impulse response of the space that the sonar
is measuring. The convolution equation can be motivated with a
range-time diagram that models the propagation of acoustic signals as
a function of range and time. Such a diagram is shown in Figure
\ref{fig:range_time_diagram_ex}.

\begin{figure}
\begin{center}
\includegraphics[width=\textwidth]{Applications/figures/rd.pdf}
\end{center}
\caption{A range-time diagram depicting the relationship between a transmitted signal and a scattered signal.}
\label{fig:range_time_diagram_ex}
\end{figure}

Modern sonars often use long waveforms in order to compress more
energy into the transmitted pulse without increasing the amount of
peak power. Longer waveforms also make it easier to separate multiple
different transmitters from one another, reducing interference. 

One of the main objectives of a sonar measurement is to recover the impulse
response $h[n]$ from measurements $m[n]$. This is trivial when the
transmit pulse is a unit impulse $x[n]=A\delta[n]$. With longer
transmit signals, we need to deconvolve the transmitted signal. This
is typically achieved by convolving the received signal $m[n]$ with a 
deconvolution filter $\lambda[n]$:
\begin{equation}
  \lambda[n]*m[n] = \lambda[n] * x[n] * h[n].
\end{equation}
The deconvolution filter needs to have the following property:
\begin{equation}
  \lambda[n]*x[n] \approx A\delta[n],
\end{equation}
where $A$ is a constant. In the case of chirp-like signals, an often used 
deconvolution filter is the time reversed chirp signal:
\begin{equation}
  \lambda[n] = x[-n].
\end{equation}

For this assignment, I have used an amplitude tapered chirp-like
waveform with a linearly increasing frequency of the following form:
\begin{equation}
  x[n] = \left\{
\begin{array}{ccc}
  \sin^{0.5}\left(\frac{\pi n}{N}\right) \sin(2\pi \beta T_s^2 n^2) & \mathrm{when} & 0 \le n < N\\
  0 & \mathrm{otherwise} & 
  \end{array}\right.
\end{equation}
Here $N$ is the integer length of the chirp pulse in samples and
$\beta = \frac{1}{2 N T_s}f_{\mathrm{max}}$ is the chirp-rate, which
determines how fast the instantaneous frequency of the signal
increases as a function of time, with $f_{\mathrm{max}}$ the maximum
frequency used of the chirp (in hertz) at the end of the transmit
pulse. Figure \ref{fig:tx_pulse_chirp} shows this waveform with
$N=10000$ and $\beta = 33075 \text{s}^{-2}$.

\begin{marginfigure}
  \begin{center}
    \includegraphics[width=\textwidth]{Assignments/figures/chirp_tx.png}
  \end{center}
  \caption{The first 2000 samples of the chirp transmit pulse signal
    $x[n]$.}
  \label{fig:tx_pulse_chirp}
\end{marginfigure}

You will need to download two data files for this task:
\begin{itemize}
  \item \url{https://github.com/jvierine/signal_processing_course/blob/main/code/027_sonar/chirp_2024.bin}
  \item \url{https://github.com/jvierine/signal_processing_course/blob/main/code/027_sonar/chirp_rec_2024.bin}
\end{itemize}
The first one contains the transmitted chirp waveform $x[n]$ and the
second one contains a microphone recording during a sonar experiment
$m[n]$.

You can read these signals into a
computer as follows:
\begin{lstlisting}[language=Python, numbers=none]
import matplotlib.pyplot as plt
import numpy as np

# Read chirp waveform.
chirp = np.fromfile("chirp_2024.bin", dtype=np.float32)
# Read microphone recording.
m = np.fromfile("chirp_rec_2024.bin", dtype=n.float32)

plt.plot(chirp)
plt.xlabel("Time (samples)")
plt.ylabel("Transmitted signal $x[n]$")
plt.show()

# Plot the first three interpulse periods.
plt.plot(m[:30000])
plt.xlabel("Time (samples)")
plt.ylabel("Received signal $m[n]$")
plt.show()
\end{lstlisting}
These acoustic signal in \verb|chirp_2024.bin| is transmitted every
$M=10000$ samples. This means that the sonar is repeating a
measurement of $h[n]$ every $10000$ samples and the transmitted signal
actually looks like this:
\begin{equation}
  \epsilon[n] = \sum_{k=0}^{N_p-1} x[n - Mk]
\end{equation}
In this task, you should treat each segment of 10000 samples as an
independent sonar measurement, which will allow you to study how the
impulse response $h[n]$ evolves over time. The first 30000 samples of the sonar echo are shown in Figure \ref{fig:rx_chirp}.
\begin{marginfigure}
  \begin{center}
    \includegraphics[width=\textwidth]{Assignments/figures/chirp_rec_2024.png}
  \end{center}
  \caption{The first 30000 samples of sonar echoes $m[n]$.}
  \label{fig:rx_chirp}
\end{marginfigure}

For this assignment, you are to perform the following tasks. Write a
short report describing what you did and what results you got. The
report is otherwise free form, as long as it is in PDF format. Include
your code and plots in the report. Please ensure that the report is
\textbf{less than three pages long}. Submit your solution to Canvas by
9.10. 12:00 if you want feedback. Keep a copy for submission at the
end of the course.

You will find a lot of help for this task in the lecture notes that
discusses LTI systems, convolution and frequency response. You may
help each other on Perusall. It is fine to give hints, but please try
not to give away the \emph{exact solution}.

\begin{enumerate}[a)]
  
\item What is the physical meaning of the signals $x[n]$ and $h[n]$ in
  Equation \ref{eq:sonar_eq_pa}?
  
\item What is the physical meaning of $r$ in Equation
  \ref{eq:sonar_eq_pa}?
  
\item How many meters of distance does an acoustic pulse travel during
  10000 samples? Assume sound speed in a typical lecture room
  at UiT (343 m/s).
  
\item How long is the transmitted sonar signal in seconds? The signal
  is in the file \verb|chirp_2024.bin|.

\item How long is the measurement in seconds? You can figure this out
  by looking at the length of the signal \verb|chirp_rec_2024.bin| and
  using the known sample-rate.
    
\item Why does the deconvolution filter $\lambda[n]$ need to have the
  property: $\lambda[n]*x[n] \approx A\delta[n]$, where $A$ is a constant?
   
\item Evaluate the autocorrelation function of the transmitted signal
  $c[n]=x[n]*x[-n]$. Use the signal that you read from
  \verb|chirp_2024.bin|. You can time-reverse a signal in Python using
  \verb|deconvolution_filter = x[::-1]|. Make a plot of $c[n]$ zooming
  into the portion of the signal that has significantly non-zero
  values. What are the implications of any deviations from a unit
  impulse for $c[n]$? Note that the function
  \verb|numpy.convolve(a,b,mode="same")| implements a convolution
  between signals $a[n]$ and $b[n]$.

%\item How close is $c[n]$ in your opinion to a scaled unit
 % impulse $A\delta[n]$? How will deviations from the unit impulse
 % affect deconvolution quality when using $\lambda[n]=x[-n]$ as a
  %deconvolution filter?
  
\item Make a plot of the scattered power as a function of time and
  range. Use one pulse for each column of the array that you are
  plotting.  You should get something similar to the plot in Figure
  \ref{fig:ex_rti_plot}. Use decibel scale for power. Use time in
  seconds on the x-axis and total propagation distance in meters on
  the y-axis. Total propagation distance means the total travelled
  distance between the loudspeaker and the microphone. You can use,
  e.g., the \verb|pcolormesh| command from Python's Matplotlib to
  make this plot. You will also find the NumPy function
  \verb|numpy.convolve(a,b,mode="same")| useful in this step.

\item Identify the strongest scatterer that is first moving away from
  the sonar, and then moving back towards the sonar. What is the
  furthest distance that the scatterer goes to before starting to move
  back towards the sonar in meters?
  
  \begin{figure}
  \begin{center}
    \includegraphics[width=\textwidth]{Assignments/figures/sonar_rti.png}
  \end{center}
  \caption{An example range-time intensity plot showing how much power
    in decibel scale is scattered back from the surroundings of the
    sonar system. The distance between the microphone and the
    loudspeaker is about 30 cm, which can be seen as a constant
    horizontal line at the propagation distance. The plot also shows
    an acoustic scatterer that is first moving away and then moving
    back towards the transmitter. Note that this is not the same plot
    that you will obtain with your data!}
  \label{fig:ex_rti_plot}
\end{figure}

\end{enumerate}
